\documentclass[11pt]{article}

    \usepackage[breakable]{tcolorbox}
    \usepackage{parskip} % Stop auto-indenting (to mimic markdown behaviour)
    

    % Basic figure setup, for now with no caption control since it's done
    % automatically by Pandoc (which extracts ![](path) syntax from Markdown).
    \usepackage{graphicx}
    % Maintain compatibility with old templates. Remove in nbconvert 6.0
    \let\Oldincludegraphics\includegraphics
    % Ensure that by default, figures have no caption (until we provide a
    % proper Figure object with a Caption API and a way to capture that
    % in the conversion process - todo).
    \usepackage{caption}
    \DeclareCaptionFormat{nocaption}{}
    \captionsetup{format=nocaption,aboveskip=0pt,belowskip=0pt}

    \usepackage{float}
    \floatplacement{figure}{H} % forces figures to be placed at the correct location
    \usepackage{xcolor} % Allow colors to be defined
    \usepackage{enumerate} % Needed for markdown enumerations to work
    \usepackage{geometry} % Used to adjust the document margins
    \usepackage{amsmath} % Equations
    \usepackage{amssymb} % Equations
    \usepackage{textcomp} % defines textquotesingle
    % Hack from http://tex.stackexchange.com/a/47451/13684:
    \AtBeginDocument{%
        \def\PYZsq{\textquotesingle}% Upright quotes in Pygmentized code
    }
    \usepackage{upquote} % Upright quotes for verbatim code
    \usepackage{eurosym} % defines \euro

    \usepackage{iftex}
    \ifPDFTeX
        \usepackage[T1]{fontenc}
        \IfFileExists{alphabeta.sty}{
              \usepackage{alphabeta}
          }{
              \usepackage[mathletters]{ucs}
              \usepackage[utf8x]{inputenc}
          }
    \else
        \usepackage{fontspec}
        \usepackage{unicode-math}
    \fi

    \usepackage{fancyvrb} % verbatim replacement that allows latex
    \usepackage{grffile} % extends the file name processing of package graphics 
                         % to support a larger range
    \makeatletter % fix for old versions of grffile with XeLaTeX
    \@ifpackagelater{grffile}{2019/11/01}
    {
      % Do nothing on new versions
    }
    {
      \def\Gread@@xetex#1{%
        \IfFileExists{"\Gin@base".bb}%
        {\Gread@eps{\Gin@base.bb}}%
        {\Gread@@xetex@aux#1}%
      }
    }
    \makeatother
    \usepackage[Export]{adjustbox} % Used to constrain images to a maximum size
    \adjustboxset{max size={0.9\linewidth}{0.9\paperheight}}

    % The hyperref package gives us a pdf with properly built
    % internal navigation ('pdf bookmarks' for the table of contents,
    % internal cross-reference links, web links for URLs, etc.)
    \usepackage{hyperref}
    % The default LaTeX title has an obnoxious amount of whitespace. By default,
    % titling removes some of it. It also provides customization options.
    \usepackage{titling}
    \usepackage{longtable} % longtable support required by pandoc >1.10
    \usepackage{booktabs}  % table support for pandoc > 1.12.2
    \usepackage{array}     % table support for pandoc >= 2.11.3
    \usepackage{calc}      % table minipage width calculation for pandoc >= 2.11.1
    \usepackage[inline]{enumitem} % IRkernel/repr support (it uses the enumerate* environment)
    \usepackage[normalem]{ulem} % ulem is needed to support strikethroughs (\sout)
                                % normalem makes italics be italics, not underlines
    \usepackage{mathrsfs}
    

    
    % Colors for the hyperref package
    \definecolor{urlcolor}{rgb}{0,.145,.698}
    \definecolor{linkcolor}{rgb}{.71,0.21,0.01}
    \definecolor{citecolor}{rgb}{.12,.54,.11}

    % ANSI colors
    \definecolor{ansi-black}{HTML}{3E424D}
    \definecolor{ansi-black-intense}{HTML}{282C36}
    \definecolor{ansi-red}{HTML}{E75C58}
    \definecolor{ansi-red-intense}{HTML}{B22B31}
    \definecolor{ansi-green}{HTML}{00A250}
    \definecolor{ansi-green-intense}{HTML}{007427}
    \definecolor{ansi-yellow}{HTML}{DDB62B}
    \definecolor{ansi-yellow-intense}{HTML}{B27D12}
    \definecolor{ansi-blue}{HTML}{208FFB}
    \definecolor{ansi-blue-intense}{HTML}{0065CA}
    \definecolor{ansi-magenta}{HTML}{D160C4}
    \definecolor{ansi-magenta-intense}{HTML}{A03196}
    \definecolor{ansi-cyan}{HTML}{60C6C8}
    \definecolor{ansi-cyan-intense}{HTML}{258F8F}
    \definecolor{ansi-white}{HTML}{C5C1B4}
    \definecolor{ansi-white-intense}{HTML}{A1A6B2}
    \definecolor{ansi-default-inverse-fg}{HTML}{FFFFFF}
    \definecolor{ansi-default-inverse-bg}{HTML}{000000}

    % common color for the border for error outputs.
    \definecolor{outerrorbackground}{HTML}{FFDFDF}

    % commands and environments needed by pandoc snippets
    % extracted from the output of `pandoc -s`
    \providecommand{\tightlist}{%
      \setlength{\itemsep}{0pt}\setlength{\parskip}{0pt}}
    \DefineVerbatimEnvironment{Highlighting}{Verbatim}{commandchars=\\\{\}}
    % Add ',fontsize=\small' for more characters per line
    \newenvironment{Shaded}{}{}
    \newcommand{\KeywordTok}[1]{\textcolor[rgb]{0.00,0.44,0.13}{\textbf{{#1}}}}
    \newcommand{\DataTypeTok}[1]{\textcolor[rgb]{0.56,0.13,0.00}{{#1}}}
    \newcommand{\DecValTok}[1]{\textcolor[rgb]{0.25,0.63,0.44}{{#1}}}
    \newcommand{\BaseNTok}[1]{\textcolor[rgb]{0.25,0.63,0.44}{{#1}}}
    \newcommand{\FloatTok}[1]{\textcolor[rgb]{0.25,0.63,0.44}{{#1}}}
    \newcommand{\CharTok}[1]{\textcolor[rgb]{0.25,0.44,0.63}{{#1}}}
    \newcommand{\StringTok}[1]{\textcolor[rgb]{0.25,0.44,0.63}{{#1}}}
    \newcommand{\CommentTok}[1]{\textcolor[rgb]{0.38,0.63,0.69}{\textit{{#1}}}}
    \newcommand{\OtherTok}[1]{\textcolor[rgb]{0.00,0.44,0.13}{{#1}}}
    \newcommand{\AlertTok}[1]{\textcolor[rgb]{1.00,0.00,0.00}{\textbf{{#1}}}}
    \newcommand{\FunctionTok}[1]{\textcolor[rgb]{0.02,0.16,0.49}{{#1}}}
    \newcommand{\RegionMarkerTok}[1]{{#1}}
    \newcommand{\ErrorTok}[1]{\textcolor[rgb]{1.00,0.00,0.00}{\textbf{{#1}}}}
    \newcommand{\NormalTok}[1]{{#1}}
    
    % Additional commands for more recent versions of Pandoc
    \newcommand{\ConstantTok}[1]{\textcolor[rgb]{0.53,0.00,0.00}{{#1}}}
    \newcommand{\SpecialCharTok}[1]{\textcolor[rgb]{0.25,0.44,0.63}{{#1}}}
    \newcommand{\VerbatimStringTok}[1]{\textcolor[rgb]{0.25,0.44,0.63}{{#1}}}
    \newcommand{\SpecialStringTok}[1]{\textcolor[rgb]{0.73,0.40,0.53}{{#1}}}
    \newcommand{\ImportTok}[1]{{#1}}
    \newcommand{\DocumentationTok}[1]{\textcolor[rgb]{0.73,0.13,0.13}{\textit{{#1}}}}
    \newcommand{\AnnotationTok}[1]{\textcolor[rgb]{0.38,0.63,0.69}{\textbf{\textit{{#1}}}}}
    \newcommand{\CommentVarTok}[1]{\textcolor[rgb]{0.38,0.63,0.69}{\textbf{\textit{{#1}}}}}
    \newcommand{\VariableTok}[1]{\textcolor[rgb]{0.10,0.09,0.49}{{#1}}}
    \newcommand{\ControlFlowTok}[1]{\textcolor[rgb]{0.00,0.44,0.13}{\textbf{{#1}}}}
    \newcommand{\OperatorTok}[1]{\textcolor[rgb]{0.40,0.40,0.40}{{#1}}}
    \newcommand{\BuiltInTok}[1]{{#1}}
    \newcommand{\ExtensionTok}[1]{{#1}}
    \newcommand{\PreprocessorTok}[1]{\textcolor[rgb]{0.74,0.48,0.00}{{#1}}}
    \newcommand{\AttributeTok}[1]{\textcolor[rgb]{0.49,0.56,0.16}{{#1}}}
    \newcommand{\InformationTok}[1]{\textcolor[rgb]{0.38,0.63,0.69}{\textbf{\textit{{#1}}}}}
    \newcommand{\WarningTok}[1]{\textcolor[rgb]{0.38,0.63,0.69}{\textbf{\textit{{#1}}}}}
    
    
    % Define a nice break command that doesn't care if a line doesn't already
    % exist.
    \def\br{\hspace*{\fill} \\* }
    % Math Jax compatibility definitions
    \def\gt{>}
    \def\lt{<}
    \let\Oldtex\TeX
    \let\Oldlatex\LaTeX
    \renewcommand{\TeX}{\textrm{\Oldtex}}
    \renewcommand{\LaTeX}{\textrm{\Oldlatex}}
    % Document parameters
    % Document title
    \title{Untitled}
    
    
    
    
    
% Pygments definitions
\makeatletter
\def\PY@reset{\let\PY@it=\relax \let\PY@bf=\relax%
    \let\PY@ul=\relax \let\PY@tc=\relax%
    \let\PY@bc=\relax \let\PY@ff=\relax}
\def\PY@tok#1{\csname PY@tok@#1\endcsname}
\def\PY@toks#1+{\ifx\relax#1\empty\else%
    \PY@tok{#1}\expandafter\PY@toks\fi}
\def\PY@do#1{\PY@bc{\PY@tc{\PY@ul{%
    \PY@it{\PY@bf{\PY@ff{#1}}}}}}}
\def\PY#1#2{\PY@reset\PY@toks#1+\relax+\PY@do{#2}}

\@namedef{PY@tok@w}{\def\PY@tc##1{\textcolor[rgb]{0.73,0.73,0.73}{##1}}}
\@namedef{PY@tok@c}{\let\PY@it=\textit\def\PY@tc##1{\textcolor[rgb]{0.24,0.48,0.48}{##1}}}
\@namedef{PY@tok@cp}{\def\PY@tc##1{\textcolor[rgb]{0.61,0.40,0.00}{##1}}}
\@namedef{PY@tok@k}{\let\PY@bf=\textbf\def\PY@tc##1{\textcolor[rgb]{0.00,0.50,0.00}{##1}}}
\@namedef{PY@tok@kp}{\def\PY@tc##1{\textcolor[rgb]{0.00,0.50,0.00}{##1}}}
\@namedef{PY@tok@kt}{\def\PY@tc##1{\textcolor[rgb]{0.69,0.00,0.25}{##1}}}
\@namedef{PY@tok@o}{\def\PY@tc##1{\textcolor[rgb]{0.40,0.40,0.40}{##1}}}
\@namedef{PY@tok@ow}{\let\PY@bf=\textbf\def\PY@tc##1{\textcolor[rgb]{0.67,0.13,1.00}{##1}}}
\@namedef{PY@tok@nb}{\def\PY@tc##1{\textcolor[rgb]{0.00,0.50,0.00}{##1}}}
\@namedef{PY@tok@nf}{\def\PY@tc##1{\textcolor[rgb]{0.00,0.00,1.00}{##1}}}
\@namedef{PY@tok@nc}{\let\PY@bf=\textbf\def\PY@tc##1{\textcolor[rgb]{0.00,0.00,1.00}{##1}}}
\@namedef{PY@tok@nn}{\let\PY@bf=\textbf\def\PY@tc##1{\textcolor[rgb]{0.00,0.00,1.00}{##1}}}
\@namedef{PY@tok@ne}{\let\PY@bf=\textbf\def\PY@tc##1{\textcolor[rgb]{0.80,0.25,0.22}{##1}}}
\@namedef{PY@tok@nv}{\def\PY@tc##1{\textcolor[rgb]{0.10,0.09,0.49}{##1}}}
\@namedef{PY@tok@no}{\def\PY@tc##1{\textcolor[rgb]{0.53,0.00,0.00}{##1}}}
\@namedef{PY@tok@nl}{\def\PY@tc##1{\textcolor[rgb]{0.46,0.46,0.00}{##1}}}
\@namedef{PY@tok@ni}{\let\PY@bf=\textbf\def\PY@tc##1{\textcolor[rgb]{0.44,0.44,0.44}{##1}}}
\@namedef{PY@tok@na}{\def\PY@tc##1{\textcolor[rgb]{0.41,0.47,0.13}{##1}}}
\@namedef{PY@tok@nt}{\let\PY@bf=\textbf\def\PY@tc##1{\textcolor[rgb]{0.00,0.50,0.00}{##1}}}
\@namedef{PY@tok@nd}{\def\PY@tc##1{\textcolor[rgb]{0.67,0.13,1.00}{##1}}}
\@namedef{PY@tok@s}{\def\PY@tc##1{\textcolor[rgb]{0.73,0.13,0.13}{##1}}}
\@namedef{PY@tok@sd}{\let\PY@it=\textit\def\PY@tc##1{\textcolor[rgb]{0.73,0.13,0.13}{##1}}}
\@namedef{PY@tok@si}{\let\PY@bf=\textbf\def\PY@tc##1{\textcolor[rgb]{0.64,0.35,0.47}{##1}}}
\@namedef{PY@tok@se}{\let\PY@bf=\textbf\def\PY@tc##1{\textcolor[rgb]{0.67,0.36,0.12}{##1}}}
\@namedef{PY@tok@sr}{\def\PY@tc##1{\textcolor[rgb]{0.64,0.35,0.47}{##1}}}
\@namedef{PY@tok@ss}{\def\PY@tc##1{\textcolor[rgb]{0.10,0.09,0.49}{##1}}}
\@namedef{PY@tok@sx}{\def\PY@tc##1{\textcolor[rgb]{0.00,0.50,0.00}{##1}}}
\@namedef{PY@tok@m}{\def\PY@tc##1{\textcolor[rgb]{0.40,0.40,0.40}{##1}}}
\@namedef{PY@tok@gh}{\let\PY@bf=\textbf\def\PY@tc##1{\textcolor[rgb]{0.00,0.00,0.50}{##1}}}
\@namedef{PY@tok@gu}{\let\PY@bf=\textbf\def\PY@tc##1{\textcolor[rgb]{0.50,0.00,0.50}{##1}}}
\@namedef{PY@tok@gd}{\def\PY@tc##1{\textcolor[rgb]{0.63,0.00,0.00}{##1}}}
\@namedef{PY@tok@gi}{\def\PY@tc##1{\textcolor[rgb]{0.00,0.52,0.00}{##1}}}
\@namedef{PY@tok@gr}{\def\PY@tc##1{\textcolor[rgb]{0.89,0.00,0.00}{##1}}}
\@namedef{PY@tok@ge}{\let\PY@it=\textit}
\@namedef{PY@tok@gs}{\let\PY@bf=\textbf}
\@namedef{PY@tok@gp}{\let\PY@bf=\textbf\def\PY@tc##1{\textcolor[rgb]{0.00,0.00,0.50}{##1}}}
\@namedef{PY@tok@go}{\def\PY@tc##1{\textcolor[rgb]{0.44,0.44,0.44}{##1}}}
\@namedef{PY@tok@gt}{\def\PY@tc##1{\textcolor[rgb]{0.00,0.27,0.87}{##1}}}
\@namedef{PY@tok@err}{\def\PY@bc##1{{\setlength{\fboxsep}{\string -\fboxrule}\fcolorbox[rgb]{1.00,0.00,0.00}{1,1,1}{\strut ##1}}}}
\@namedef{PY@tok@kc}{\let\PY@bf=\textbf\def\PY@tc##1{\textcolor[rgb]{0.00,0.50,0.00}{##1}}}
\@namedef{PY@tok@kd}{\let\PY@bf=\textbf\def\PY@tc##1{\textcolor[rgb]{0.00,0.50,0.00}{##1}}}
\@namedef{PY@tok@kn}{\let\PY@bf=\textbf\def\PY@tc##1{\textcolor[rgb]{0.00,0.50,0.00}{##1}}}
\@namedef{PY@tok@kr}{\let\PY@bf=\textbf\def\PY@tc##1{\textcolor[rgb]{0.00,0.50,0.00}{##1}}}
\@namedef{PY@tok@bp}{\def\PY@tc##1{\textcolor[rgb]{0.00,0.50,0.00}{##1}}}
\@namedef{PY@tok@fm}{\def\PY@tc##1{\textcolor[rgb]{0.00,0.00,1.00}{##1}}}
\@namedef{PY@tok@vc}{\def\PY@tc##1{\textcolor[rgb]{0.10,0.09,0.49}{##1}}}
\@namedef{PY@tok@vg}{\def\PY@tc##1{\textcolor[rgb]{0.10,0.09,0.49}{##1}}}
\@namedef{PY@tok@vi}{\def\PY@tc##1{\textcolor[rgb]{0.10,0.09,0.49}{##1}}}
\@namedef{PY@tok@vm}{\def\PY@tc##1{\textcolor[rgb]{0.10,0.09,0.49}{##1}}}
\@namedef{PY@tok@sa}{\def\PY@tc##1{\textcolor[rgb]{0.73,0.13,0.13}{##1}}}
\@namedef{PY@tok@sb}{\def\PY@tc##1{\textcolor[rgb]{0.73,0.13,0.13}{##1}}}
\@namedef{PY@tok@sc}{\def\PY@tc##1{\textcolor[rgb]{0.73,0.13,0.13}{##1}}}
\@namedef{PY@tok@dl}{\def\PY@tc##1{\textcolor[rgb]{0.73,0.13,0.13}{##1}}}
\@namedef{PY@tok@s2}{\def\PY@tc##1{\textcolor[rgb]{0.73,0.13,0.13}{##1}}}
\@namedef{PY@tok@sh}{\def\PY@tc##1{\textcolor[rgb]{0.73,0.13,0.13}{##1}}}
\@namedef{PY@tok@s1}{\def\PY@tc##1{\textcolor[rgb]{0.73,0.13,0.13}{##1}}}
\@namedef{PY@tok@mb}{\def\PY@tc##1{\textcolor[rgb]{0.40,0.40,0.40}{##1}}}
\@namedef{PY@tok@mf}{\def\PY@tc##1{\textcolor[rgb]{0.40,0.40,0.40}{##1}}}
\@namedef{PY@tok@mh}{\def\PY@tc##1{\textcolor[rgb]{0.40,0.40,0.40}{##1}}}
\@namedef{PY@tok@mi}{\def\PY@tc##1{\textcolor[rgb]{0.40,0.40,0.40}{##1}}}
\@namedef{PY@tok@il}{\def\PY@tc##1{\textcolor[rgb]{0.40,0.40,0.40}{##1}}}
\@namedef{PY@tok@mo}{\def\PY@tc##1{\textcolor[rgb]{0.40,0.40,0.40}{##1}}}
\@namedef{PY@tok@ch}{\let\PY@it=\textit\def\PY@tc##1{\textcolor[rgb]{0.24,0.48,0.48}{##1}}}
\@namedef{PY@tok@cm}{\let\PY@it=\textit\def\PY@tc##1{\textcolor[rgb]{0.24,0.48,0.48}{##1}}}
\@namedef{PY@tok@cpf}{\let\PY@it=\textit\def\PY@tc##1{\textcolor[rgb]{0.24,0.48,0.48}{##1}}}
\@namedef{PY@tok@c1}{\let\PY@it=\textit\def\PY@tc##1{\textcolor[rgb]{0.24,0.48,0.48}{##1}}}
\@namedef{PY@tok@cs}{\let\PY@it=\textit\def\PY@tc##1{\textcolor[rgb]{0.24,0.48,0.48}{##1}}}

\def\PYZbs{\char`\\}
\def\PYZus{\char`\_}
\def\PYZob{\char`\{}
\def\PYZcb{\char`\}}
\def\PYZca{\char`\^}
\def\PYZam{\char`\&}
\def\PYZlt{\char`\<}
\def\PYZgt{\char`\>}
\def\PYZsh{\char`\#}
\def\PYZpc{\char`\%}
\def\PYZdl{\char`\$}
\def\PYZhy{\char`\-}
\def\PYZsq{\char`\'}
\def\PYZdq{\char`\"}
\def\PYZti{\char`\~}
% for compatibility with earlier versions
\def\PYZat{@}
\def\PYZlb{[}
\def\PYZrb{]}
\makeatother


    % For linebreaks inside Verbatim environment from package fancyvrb. 
    \makeatletter
        \newbox\Wrappedcontinuationbox 
        \newbox\Wrappedvisiblespacebox 
        \newcommand*\Wrappedvisiblespace {\textcolor{red}{\textvisiblespace}} 
        \newcommand*\Wrappedcontinuationsymbol {\textcolor{red}{\llap{\tiny$\m@th\hookrightarrow$}}} 
        \newcommand*\Wrappedcontinuationindent {3ex } 
        \newcommand*\Wrappedafterbreak {\kern\Wrappedcontinuationindent\copy\Wrappedcontinuationbox} 
        % Take advantage of the already applied Pygments mark-up to insert 
        % potential linebreaks for TeX processing. 
        %        {, <, #, %, $, ' and ": go to next line. 
        %        _, }, ^, &, >, - and ~: stay at end of broken line. 
        % Use of \textquotesingle for straight quote. 
        \newcommand*\Wrappedbreaksatspecials {% 
            \def\PYGZus{\discretionary{\char`\_}{\Wrappedafterbreak}{\char`\_}}% 
            \def\PYGZob{\discretionary{}{\Wrappedafterbreak\char`\{}{\char`\{}}% 
            \def\PYGZcb{\discretionary{\char`\}}{\Wrappedafterbreak}{\char`\}}}% 
            \def\PYGZca{\discretionary{\char`\^}{\Wrappedafterbreak}{\char`\^}}% 
            \def\PYGZam{\discretionary{\char`\&}{\Wrappedafterbreak}{\char`\&}}% 
            \def\PYGZlt{\discretionary{}{\Wrappedafterbreak\char`\<}{\char`\<}}% 
            \def\PYGZgt{\discretionary{\char`\>}{\Wrappedafterbreak}{\char`\>}}% 
            \def\PYGZsh{\discretionary{}{\Wrappedafterbreak\char`\#}{\char`\#}}% 
            \def\PYGZpc{\discretionary{}{\Wrappedafterbreak\char`\%}{\char`\%}}% 
            \def\PYGZdl{\discretionary{}{\Wrappedafterbreak\char`\$}{\char`\$}}% 
            \def\PYGZhy{\discretionary{\char`\-}{\Wrappedafterbreak}{\char`\-}}% 
            \def\PYGZsq{\discretionary{}{\Wrappedafterbreak\textquotesingle}{\textquotesingle}}% 
            \def\PYGZdq{\discretionary{}{\Wrappedafterbreak\char`\"}{\char`\"}}% 
            \def\PYGZti{\discretionary{\char`\~}{\Wrappedafterbreak}{\char`\~}}% 
        } 
        % Some characters . , ; ? ! / are not pygmentized. 
        % This macro makes them "active" and they will insert potential linebreaks 
        \newcommand*\Wrappedbreaksatpunct {% 
            \lccode`\~`\.\lowercase{\def~}{\discretionary{\hbox{\char`\.}}{\Wrappedafterbreak}{\hbox{\char`\.}}}% 
            \lccode`\~`\,\lowercase{\def~}{\discretionary{\hbox{\char`\,}}{\Wrappedafterbreak}{\hbox{\char`\,}}}% 
            \lccode`\~`\;\lowercase{\def~}{\discretionary{\hbox{\char`\;}}{\Wrappedafterbreak}{\hbox{\char`\;}}}% 
            \lccode`\~`\:\lowercase{\def~}{\discretionary{\hbox{\char`\:}}{\Wrappedafterbreak}{\hbox{\char`\:}}}% 
            \lccode`\~`\?\lowercase{\def~}{\discretionary{\hbox{\char`\?}}{\Wrappedafterbreak}{\hbox{\char`\?}}}% 
            \lccode`\~`\!\lowercase{\def~}{\discretionary{\hbox{\char`\!}}{\Wrappedafterbreak}{\hbox{\char`\!}}}% 
            \lccode`\~`\/\lowercase{\def~}{\discretionary{\hbox{\char`\/}}{\Wrappedafterbreak}{\hbox{\char`\/}}}% 
            \catcode`\.\active
            \catcode`\,\active 
            \catcode`\;\active
            \catcode`\:\active
            \catcode`\?\active
            \catcode`\!\active
            \catcode`\/\active 
            \lccode`\~`\~ 	
        }
    \makeatother

    \let\OriginalVerbatim=\Verbatim
    \makeatletter
    \renewcommand{\Verbatim}[1][1]{%
        %\parskip\z@skip
        \sbox\Wrappedcontinuationbox {\Wrappedcontinuationsymbol}%
        \sbox\Wrappedvisiblespacebox {\FV@SetupFont\Wrappedvisiblespace}%
        \def\FancyVerbFormatLine ##1{\hsize\linewidth
            \vtop{\raggedright\hyphenpenalty\z@\exhyphenpenalty\z@
                \doublehyphendemerits\z@\finalhyphendemerits\z@
                \strut ##1\strut}%
        }%
        % If the linebreak is at a space, the latter will be displayed as visible
        % space at end of first line, and a continuation symbol starts next line.
        % Stretch/shrink are however usually zero for typewriter font.
        \def\FV@Space {%
            \nobreak\hskip\z@ plus\fontdimen3\font minus\fontdimen4\font
            \discretionary{\copy\Wrappedvisiblespacebox}{\Wrappedafterbreak}
            {\kern\fontdimen2\font}%
        }%
        
        % Allow breaks at special characters using \PYG... macros.
        \Wrappedbreaksatspecials
        % Breaks at punctuation characters . , ; ? ! and / need catcode=\active 	
        \OriginalVerbatim[#1,codes*=\Wrappedbreaksatpunct]%
    }
    \makeatother

    % Exact colors from NB
    \definecolor{incolor}{HTML}{303F9F}
    \definecolor{outcolor}{HTML}{D84315}
    \definecolor{cellborder}{HTML}{CFCFCF}
    \definecolor{cellbackground}{HTML}{F7F7F7}
    
    % prompt
    \makeatletter
    \newcommand{\boxspacing}{\kern\kvtcb@left@rule\kern\kvtcb@boxsep}
    \makeatother
    \newcommand{\prompt}[4]{
        {\ttfamily\llap{{\color{#2}[#3]:\hspace{3pt}#4}}\vspace{-\baselineskip}}
    }
    

    
    % Prevent overflowing lines due to hard-to-break entities
    \sloppy 
    % Setup hyperref package
    \hypersetup{
      breaklinks=true,  % so long urls are correctly broken across lines
      colorlinks=true,
      urlcolor=urlcolor,
      linkcolor=linkcolor,
      citecolor=citecolor,
      }
    % Slightly bigger margins than the latex defaults
    
    \geometry{verbose,tmargin=1in,bmargin=1in,lmargin=1in,rmargin=1in}
    
    

\begin{document}
    
    \maketitle
    
    

    
    Answer to theoretical questions:

Magnetization operator:
\(m(\phi)=\frac{\sinh(\beta h\pm\phi)}{\cosh(\beta \pm\phi)}=\tanh(\beta h+\phi)\)

Energy per site operator:
\(\epsilon(\phi)=-\frac{\phi^2-J\beta}{J\beta^2}\frac1{2N}-h\tanh(\beta h+\phi)\)

\(\dot{\phi}=p\)

\(\dot{p}=-\frac{\phi}{\beta J}+N\cdot \tanh{(\beta\cdot h+\phi)}\)

    \begin{tcolorbox}[breakable, size=fbox, boxrule=1pt, pad at break*=1mm,colback=cellbackground, colframe=cellborder]
\prompt{In}{incolor}{33}{\boxspacing}
\begin{Verbatim}[commandchars=\\\{\}]
\PY{k}{def} \PY{n+nf}{leapfrog\PYZus{}plot}\PY{p}{(}\PY{p}{)}\PY{p}{:}
    \PY{k}{global} \PY{n}{N\PYZus{}md}
    \PY{n}{p\PYZus{}0}\PY{o}{=}\PY{l+m+mf}{0.5}
    \PY{n}{phi\PYZus{}0}\PY{o}{=}\PY{l+m+mf}{.5}
    \PY{n}{H\PYZus{}0}\PY{o}{=}\PY{n}{H}\PY{p}{(}\PY{n}{p\PYZus{}0}\PY{p}{,}\PY{n}{phi\PYZus{}0}\PY{p}{)}
    \PY{k}{for} \PY{n}{i} \PY{o+ow}{in} \PY{n+nb}{range}\PY{p}{(}\PY{l+m+mi}{100}\PY{p}{)}\PY{p}{:}
        \PY{n}{N\PYZus{}md}\PY{o}{=}\PY{n}{i}\PY{o}{+}\PY{l+m+mi}{2}
        \PY{n}{p}\PY{p}{,}\PY{n}{phi}\PY{o}{=}\PY{n}{leapfrog}\PY{p}{(}\PY{n}{p\PYZus{}0}\PY{p}{,}\PY{n}{phi\PYZus{}0}\PY{p}{)}
        \PY{n}{plt}\PY{o}{.}\PY{n}{plot}\PY{p}{(}\PY{n}{i}\PY{o}{+}\PY{l+m+mi}{2}\PY{p}{,}\PY{n+nb}{abs}\PY{p}{(}\PY{p}{(}\PY{n}{H}\PY{p}{(}\PY{n}{p}\PY{p}{,}\PY{n}{phi}\PY{p}{)}\PY{o}{\PYZhy{}}\PY{n}{H\PYZus{}0}\PY{p}{)}\PY{o}{/}\PY{n}{H\PYZus{}0}\PY{p}{)}\PY{p}{,} \PY{l+s+s1}{\PYZsq{}}\PY{l+s+s1}{x}\PY{l+s+s1}{\PYZsq{}}\PY{p}{,} \PY{n}{color}\PY{o}{=}\PY{l+s+s1}{\PYZsq{}}\PY{l+s+s1}{b}\PY{l+s+s1}{\PYZsq{}}\PY{p}{)}
    \PY{n}{plt}\PY{o}{.}\PY{n}{semilogy}\PY{p}{(}\PY{p}{)}    
    \PY{n}{plt}\PY{o}{.}\PY{n}{show}\PY{p}{(}\PY{p}{)} 
\end{Verbatim}
\end{tcolorbox}

    \begin{tcolorbox}[breakable, size=fbox, boxrule=1pt, pad at break*=1mm,colback=cellbackground, colframe=cellborder]
\prompt{In}{incolor}{34}{\boxspacing}
\begin{Verbatim}[commandchars=\\\{\}]
\PY{n}{leapfrog\PYZus{}plot}\PY{p}{(}\PY{p}{)}
\end{Verbatim}
\end{tcolorbox}

    \begin{center}
    \adjustimage{max size={0.9\linewidth}{0.9\paperheight}}{output_2_0.png}
    \end{center}
    { \hspace*{\fill} \\}
    
    The graph looks like we expected it. It shows that for large \(N_{md}\)
the energy of the old and new configuration is almost the same. This
will later be important to achieve a high acceptance probability.

    \begin{tcolorbox}[breakable, size=fbox, boxrule=1pt, pad at break*=1mm,colback=cellbackground, colframe=cellborder]
\prompt{In}{incolor}{35}{\boxspacing}
\begin{Verbatim}[commandchars=\\\{\}]
\PY{c+ch}{\PYZsh{}!/usr/bin/env python}
\PY{c+c1}{\PYZsh{} coding: utf\PYZhy{}8}

\PY{k+kn}{import} \PY{n+nn}{numpy} \PY{k}{as} \PY{n+nn}{np}
\PY{k+kn}{from}  \PY{n+nn}{math} \PY{k+kn}{import} \PY{o}{*}
\PY{k+kn}{import} \PY{n+nn}{matplotlib}\PY{n+nn}{.}\PY{n+nn}{pyplot} \PY{k}{as} \PY{n+nn}{plt}
\PY{k+kn}{import} \PY{n+nn}{scipy}\PY{n+nn}{.}\PY{n+nn}{optimize} \PY{k}{as} \PY{n+nn}{so}
\PY{k+kn}{import} \PY{n+nn}{scipy}\PY{n+nn}{.}\PY{n+nn}{special} \PY{k}{as} \PY{n+nn}{sp}
\PY{k+kn}{import} \PY{n+nn}{mpmath} \PY{k}{as} \PY{n+nn}{mp}


\PY{n}{N\PYZus{}md} \PY{o}{=} \PY{l+m+mi}{50} \PY{c+c1}{\PYZsh{}Leapfrog integration steps}
\PY{n}{N\PYZus{}en} \PY{o}{=} \PY{l+m+mi}{10}
\PY{n}{N\PYZus{}cfg} \PY{o}{=} \PY{l+m+mi}{15000}
\PY{n}{beta}\PY{o}{=}\PY{l+m+mi}{1}
\PY{n}{N}\PY{o}{=}\PY{l+m+mi}{20}
\PY{n}{h}\PY{o}{=}\PY{l+m+mf}{0.5}
\PY{n}{beta\PYZus{}h}\PY{o}{=}\PY{l+m+mf}{0.5}
\PY{n}{J}\PY{o}{=}\PY{l+m+mi}{1}\PY{o}{/}\PY{n}{N}
\PY{n}{phi\PYZus{}0} \PY{o}{=} \PY{l+m+mi}{0}
\PY{n}{p\PYZus{}0} \PY{o}{=} \PY{l+m+mi}{0} 
\PY{n}{I} \PY{o}{=} \PY{l+m+mi}{10}
\PY{n}{A} \PY{o}{=} \PY{l+m+mi}{1} 
\PY{n}{ar}\PY{o}{=}\PY{l+m+mi}{0}
\end{Verbatim}
\end{tcolorbox}

    This is just defining of variables.

    \begin{tcolorbox}[breakable, size=fbox, boxrule=1pt, pad at break*=1mm,colback=cellbackground, colframe=cellborder]
\prompt{In}{incolor}{36}{\boxspacing}
\begin{Verbatim}[commandchars=\\\{\}]
\PY{k}{def} \PY{n+nf}{calc\PYZus{}magn}\PY{p}{(}\PY{n}{phi\PYZus{}en}\PY{p}{)}\PY{p}{:} 
    \PY{k}{global} \PY{n}{beta\PYZus{}h}
    \PY{k}{global} \PY{n}{N\PYZus{}cfg}
    \PY{k}{global} \PY{n}{N\PYZus{}en}
    \PY{k}{global} \PY{n}{I}
    \PY{k}{global} \PY{n}{A}
    
    \PY{n}{M\PYZus{}array} \PY{o}{=} \PY{n}{np}\PY{o}{.}\PY{n}{zeros}\PY{p}{(}\PY{n}{N\PYZus{}cfg}\PY{p}{)}
    
    \PY{k}{for} \PY{n}{i} \PY{o+ow}{in} \PY{n+nb}{range}\PY{p}{(}\PY{n}{N\PYZus{}cfg}\PY{p}{)}\PY{p}{:}
        \PY{n}{M\PYZus{}array}\PY{p}{[}\PY{n}{i}\PY{p}{]} \PY{o}{=} \PY{n}{np}\PY{o}{.}\PY{n}{tanh}\PY{p}{(}\PY{n}{beta\PYZus{}h}\PY{o}{+}\PY{n}{phi\PYZus{}en}\PY{p}{[}\PY{n}{i}\PY{p}{]}\PY{p}{)} 
      
    \PY{n}{M\PYZus{}array} \PY{o}{=} \PY{n}{M\PYZus{}array}\PY{p}{[}\PY{n}{I}\PY{p}{:}\PY{p}{]}
    \PY{n}{M}\PY{o}{=}\PY{n}{np}\PY{o}{.}\PY{n}{sum}\PY{p}{(}\PY{n}{M\PYZus{}array}\PY{p}{)}
    
    \PY{k}{return} \PY{n}{M}\PY{o}{/}\PY{p}{(}\PY{n}{N\PYZus{}cfg}\PY{p}{)}
\end{Verbatim}
\end{tcolorbox}

    input: Array with values of \(\phi\) for an entire ensemble; output:
Mean Magnetization of the ensemble\\
In the beginning of the array we cut away some configurations for
thermalization.

    \begin{tcolorbox}[breakable, size=fbox, boxrule=1pt, pad at break*=1mm,colback=cellbackground, colframe=cellborder]
\prompt{In}{incolor}{37}{\boxspacing}
\begin{Verbatim}[commandchars=\\\{\}]
\PY{k}{def} \PY{n+nf}{calc\PYZus{}eps}\PY{p}{(}\PY{n}{phi\PYZus{}en}\PY{p}{)}\PY{p}{:} 
    \PY{k}{global} \PY{n}{beta\PYZus{}h}
    \PY{k}{global} \PY{n}{h}
    \PY{k}{global} \PY{n}{beta}
    \PY{k}{global} \PY{n}{N\PYZus{}cfg}
    \PY{k}{global} \PY{n}{J}
    \PY{k}{global} \PY{n}{I}
    
    \PY{n}{e\PYZus{}array} \PY{o}{=} \PY{n}{np}\PY{o}{.}\PY{n}{zeros}\PY{p}{(}\PY{n}{N\PYZus{}cfg}\PY{p}{)}
        
    \PY{k}{for} \PY{n}{i} \PY{o+ow}{in} \PY{n+nb}{range}\PY{p}{(}\PY{n}{N\PYZus{}cfg}\PY{p}{)}\PY{p}{:}
        \PY{n}{e\PYZus{}array}\PY{p}{[}\PY{n}{i}\PY{p}{]}\PY{o}{=} \PY{p}{(}\PY{n}{phi\PYZus{}en}\PY{p}{[}\PY{n}{i}\PY{p}{]}\PY{o}{*}\PY{o}{*}\PY{l+m+mi}{2}\PY{o}{\PYZhy{}}\PY{n}{J}\PY{o}{*}\PY{n}{beta}\PY{p}{)}\PY{o}{/}\PY{p}{(}\PY{n}{J}\PY{o}{*}\PY{n}{beta}\PY{o}{*}\PY{o}{*}\PY{l+m+mi}{2}\PY{p}{)}\PY{o}{/}\PY{p}{(}\PY{l+m+mi}{2}\PY{o}{*}\PY{n}{N}\PY{p}{)}\PY{o}{+}\PY{n}{h}\PY{o}{*}\PY{n}{mp}\PY{o}{.}\PY{n}{tanh}\PY{p}{(}\PY{n}{beta\PYZus{}h}\PY{o}{+}\PY{n}{phi\PYZus{}en}\PY{p}{[}\PY{n}{i}\PY{p}{]}\PY{p}{)}
    \PY{n}{e\PYZus{}array}\PY{o}{=}\PY{n}{e\PYZus{}array}\PY{p}{[}\PY{n}{I}\PY{p}{:}\PY{p}{]}
    \PY{n}{e} \PY{o}{=} \PY{n}{np}\PY{o}{.}\PY{n}{sum}\PY{p}{(}\PY{n}{e\PYZus{}array}\PY{p}{)}
    
    \PY{k}{return} \PY{o}{\PYZhy{}}\PY{n}{e}\PY{o}{/}\PY{n}{N\PYZus{}cfg}
\end{Verbatim}
\end{tcolorbox}

    input: Array with values of \(\phi\) for an entire ensemble; output:
Mean energy per site of the ensemble\\
In the beginning of the array we cut away some configurations for
thermalization.

    \begin{tcolorbox}[breakable, size=fbox, boxrule=1pt, pad at break*=1mm,colback=cellbackground, colframe=cellborder]
\prompt{In}{incolor}{38}{\boxspacing}
\begin{Verbatim}[commandchars=\\\{\}]
\PY{k}{def} \PY{n+nf}{f}\PY{p}{(}\PY{n}{n}\PY{p}{)}\PY{p}{:}
    \PY{k}{global} \PY{n}{J}
    \PY{k}{global} \PY{n}{beta\PYZus{}J}
    \PY{k}{global} \PY{n}{beta\PYZus{}h}
    \PY{k}{global} \PY{n}{beta} 
      
   
    \PY{k}{return}\PY{p}{(}\PY{n}{np}\PY{o}{.}\PY{n}{exp}\PY{p}{(}\PY{p}{(}\PY{n}{beta}\PY{o}{*}\PY{n}{J}\PY{o}{*}\PY{n}{n}\PY{o}{*}\PY{o}{*}\PY{l+m+mi}{2}\PY{p}{)}\PY{o}{/}\PY{l+m+mi}{2}\PY{o}{+}\PY{n}{beta\PYZus{}h}\PY{o}{*}\PY{n}{n}\PY{p}{)}\PY{p}{)}
\end{Verbatim}
\end{tcolorbox}

    input: x; output: f(x)

    \begin{tcolorbox}[breakable, size=fbox, boxrule=1pt, pad at break*=1mm,colback=cellbackground, colframe=cellborder]
\prompt{In}{incolor}{39}{\boxspacing}
\begin{Verbatim}[commandchars=\\\{\}]
\PY{k}{def} \PY{n+nf}{partition\PYZus{}func\PYZus{}theo}\PY{p}{(}\PY{p}{)}\PY{p}{:}
    \PY{k}{global} \PY{n}{beta\PYZus{}h}
    \PY{k}{global} \PY{n}{N}
    \PY{k}{global} \PY{n}{beta\PYZus{}J}
    
    \PY{n}{Z} \PY{o}{=} \PY{l+m+mi}{0}
   
    
    \PY{k}{for} \PY{n}{n} \PY{o+ow}{in} \PY{n+nb}{range}\PY{p}{(}\PY{n}{N}\PY{o}{+}\PY{l+m+mi}{1}\PY{p}{)}\PY{p}{:}
        \PY{n}{Z} \PY{o}{+}\PY{o}{=} \PY{n}{sp}\PY{o}{.}\PY{n}{binom}\PY{p}{(}\PY{n}{N}\PY{p}{,} \PY{n}{n}\PY{p}{)}\PY{o}{*}\PY{n}{f}\PY{p}{(}\PY{n}{N}\PY{o}{\PYZhy{}}\PY{l+m+mi}{2}\PY{o}{*}\PY{n}{n}\PY{p}{)}
    
    \PY{k}{return}\PY{p}{(}\PY{n}{Z}\PY{p}{)}
\end{Verbatim}
\end{tcolorbox}

    input: ; output: Z(N,beta,J,h)

    \begin{tcolorbox}[breakable, size=fbox, boxrule=1pt, pad at break*=1mm,colback=cellbackground, colframe=cellborder]
\prompt{In}{incolor}{40}{\boxspacing}
\begin{Verbatim}[commandchars=\\\{\}]
\PY{k}{def} \PY{n+nf}{beta\PYZus{}eps\PYZus{}theo}\PY{p}{(}\PY{p}{)}\PY{p}{:}
    \PY{k}{global} \PY{n}{N}
    \PY{k}{global} \PY{n}{beta\PYZus{}J}
    \PY{k}{global} \PY{n}{beta\PYZus{}h}
    
    \PY{n}{value} \PY{o}{=} \PY{l+m+mi}{0}
    
    
    \PY{k}{for} \PY{n}{n} \PY{o+ow}{in} \PY{n+nb}{range}\PY{p}{(}\PY{n}{N}\PY{o}{+}\PY{l+m+mi}{1}\PY{p}{)}\PY{p}{:}
        
        \PY{n}{value} \PY{o}{+}\PY{o}{=} \PY{n}{sp}\PY{o}{.}\PY{n}{binom}\PY{p}{(}\PY{n}{N}\PY{p}{,} \PY{n}{n}\PY{p}{)}\PY{o}{*}\PY{p}{(}\PY{l+m+mi}{1}\PY{o}{/}\PY{l+m+mi}{2}\PY{o}{*}\PY{n}{beta}\PY{o}{*}\PY{n}{J}\PY{o}{*}\PY{p}{(}\PY{n}{N}\PY{o}{\PYZhy{}}\PY{l+m+mi}{2}\PY{o}{*}\PY{n}{n}\PY{p}{)}\PY{o}{*}\PY{o}{*}\PY{l+m+mi}{2}\PY{o}{+}\PY{n}{beta\PYZus{}h}\PY{o}{*}\PY{p}{(}\PY{n}{N}\PY{o}{\PYZhy{}}\PY{l+m+mi}{2}\PY{o}{*}\PY{n}{n}\PY{p}{)}\PY{p}{)}\PY{o}{*}\PY{n}{f}\PY{p}{(}\PY{n}{N}\PY{o}{\PYZhy{}}\PY{l+m+mi}{2}\PY{o}{*}\PY{n}{n}\PY{p}{)}
    
    \PY{n}{value} \PY{o}{=} \PY{o}{\PYZhy{}}\PY{n}{value}\PY{o}{/}\PY{p}{(}\PY{n}{N}\PY{o}{*}\PY{n}{partition\PYZus{}func\PYZus{}theo}\PY{p}{(}\PY{p}{)}\PY{p}{)}
    
    \PY{k}{return}\PY{p}{(}\PY{n}{value}\PY{p}{)}
\end{Verbatim}
\end{tcolorbox}

    input: ; output: beta*eps(N,beta,J,h)

    \begin{tcolorbox}[breakable, size=fbox, boxrule=1pt, pad at break*=1mm,colback=cellbackground, colframe=cellborder]
\prompt{In}{incolor}{41}{\boxspacing}
\begin{Verbatim}[commandchars=\\\{\}]
\PY{k}{def} \PY{n+nf}{m\PYZus{}theo}\PY{p}{(}\PY{p}{)}\PY{p}{:}
    \PY{k}{global} \PY{n}{N}
    \PY{k}{global} \PY{n}{beta\PYZus{}J}
    \PY{k}{global} \PY{n}{beta\PYZus{}h}
    \PY{k}{global} \PY{n}{J}
    
    \PY{n}{value} \PY{o}{=} \PY{l+m+mi}{0}
    
    \PY{k}{for} \PY{n}{n} \PY{o+ow}{in} \PY{n+nb}{range}\PY{p}{(}\PY{n}{N}\PY{o}{+}\PY{l+m+mi}{1}\PY{p}{)}\PY{p}{:}
        \PY{n}{value} \PY{o}{+}\PY{o}{=} \PY{n}{sp}\PY{o}{.}\PY{n}{binom}\PY{p}{(}\PY{n}{N}\PY{p}{,} \PY{n}{n}\PY{p}{)}\PY{o}{*}\PY{p}{(}\PY{n}{N}\PY{o}{\PYZhy{}}\PY{l+m+mi}{2}\PY{o}{*}\PY{n}{n}\PY{p}{)}\PY{o}{*}\PY{n}{f}\PY{p}{(}\PY{n}{N}\PY{o}{\PYZhy{}}\PY{l+m+mi}{2}\PY{o}{*}\PY{n}{n}\PY{p}{)}
        
    \PY{n}{value} \PY{o}{=} \PY{n}{value}\PY{o}{/}\PY{p}{(}\PY{n}{N}\PY{o}{*}\PY{n}{partition\PYZus{}func\PYZus{}theo}\PY{p}{(}\PY{p}{)}\PY{p}{)}
    \PY{k}{return}\PY{p}{(}\PY{n}{value}\PY{p}{)}
\end{Verbatim}
\end{tcolorbox}

    input: ; output: m(N,beta,J,h)

    \begin{tcolorbox}[breakable, size=fbox, boxrule=1pt, pad at break*=1mm,colback=cellbackground, colframe=cellborder]
\prompt{In}{incolor}{42}{\boxspacing}
\begin{Verbatim}[commandchars=\\\{\}]
\PY{k}{def} \PY{n+nf}{leapfrog}\PY{p}{(}\PY{n}{p\PYZus{}0}\PY{p}{,}\PY{n}{phi\PYZus{}0}\PY{p}{)}\PY{p}{:}
    \PY{k}{global} \PY{n}{beta} 
    \PY{k}{global} \PY{n}{J}
    \PY{k}{global} \PY{n}{N} 
    \PY{k}{global} \PY{n}{h}
    \PY{k}{global} \PY{n}{N\PYZus{}md}
    
    \PY{n}{eps}\PY{o}{=}\PY{l+m+mi}{1}\PY{o}{/}\PY{n}{N\PYZus{}md}
    \PY{n}{phi}\PY{o}{=}\PY{n}{phi\PYZus{}0}\PY{o}{+}\PY{n}{eps}\PY{o}{/}\PY{l+m+mi}{2}\PY{o}{*}\PY{n}{p\PYZus{}0}
    \PY{n}{p\PYZus{}leap}\PY{o}{=}\PY{n}{p\PYZus{}0}
    \PY{k}{for} \PY{n}{i} \PY{o+ow}{in} \PY{n+nb}{range}\PY{p}{(}\PY{n}{N\PYZus{}md}\PY{o}{\PYZhy{}}\PY{l+m+mi}{1}\PY{p}{)}\PY{p}{:}
        \PY{n}{p\PYZus{}leap}\PY{o}{=}\PY{n}{p\PYZus{}leap}\PY{o}{\PYZhy{}}\PY{n}{eps}\PY{o}{*}\PY{p}{(}\PY{n}{phi}\PY{o}{/}\PY{p}{(}\PY{n}{beta}\PY{o}{*}\PY{n}{J}\PY{p}{)}\PY{o}{\PYZhy{}}\PY{n}{N}\PY{o}{*}\PY{n}{mp}\PY{o}{.}\PY{n}{tanh}\PY{p}{(}\PY{n}{beta}\PY{o}{*}\PY{n}{h}\PY{o}{+}\PY{n}{phi}\PY{p}{)}\PY{p}{)}
        \PY{n}{phi}\PY{o}{=}\PY{n}{phi}\PY{o}{+}\PY{n}{eps}\PY{o}{*}\PY{n}{p\PYZus{}leap}
    \PY{n}{p\PYZus{}leap}\PY{o}{=}\PY{n}{p\PYZus{}leap}\PY{o}{\PYZhy{}}\PY{n}{eps}\PY{o}{*}\PY{p}{(}\PY{n}{phi}\PY{o}{/}\PY{p}{(}\PY{n}{beta}\PY{o}{*}\PY{n}{J}\PY{p}{)}\PY{o}{\PYZhy{}}\PY{n}{N}\PY{o}{*}\PY{n}{mp}\PY{o}{.}\PY{n}{tanh}\PY{p}{(}\PY{n}{beta}\PY{o}{*}\PY{n}{h}\PY{o}{+}\PY{n}{phi}\PY{p}{)}\PY{p}{)}
    \PY{n}{phi}\PY{o}{=}\PY{n}{phi}\PY{o}{+}\PY{n}{eps}\PY{o}{/}\PY{l+m+mi}{2}\PY{o}{*}\PY{n}{p\PYZus{}leap}
    
    \PY{k}{return} \PY{n}{p\PYZus{}leap}\PY{p}{,}\PY{n}{phi}
\end{Verbatim}
\end{tcolorbox}

    input: p\_0, phi\_0; output: p\_f,phi\_f\\
code as explained on the sheet

    \begin{tcolorbox}[breakable, size=fbox, boxrule=1pt, pad at break*=1mm,colback=cellbackground, colframe=cellborder]
\prompt{In}{incolor}{43}{\boxspacing}
\begin{Verbatim}[commandchars=\\\{\}]
\PY{k}{def} \PY{n+nf}{H}\PY{p}{(}\PY{n}{p}\PY{p}{,}\PY{n}{phi}\PY{p}{)}\PY{p}{:}
    \PY{k}{global} \PY{n}{beta} 
    \PY{k}{global} \PY{n}{J} 
    \PY{k}{global} \PY{n}{h} 
    
    \PY{k}{return} \PY{n}{p}\PY{o}{*}\PY{o}{*}\PY{l+m+mi}{2}\PY{o}{/}\PY{l+m+mi}{2}\PY{o}{+}\PY{n}{phi}\PY{o}{*}\PY{o}{*}\PY{l+m+mi}{2}\PY{o}{/}\PY{p}{(}\PY{l+m+mi}{2}\PY{o}{*}\PY{n}{beta}\PY{o}{*}\PY{n}{J}\PY{p}{)}\PY{o}{\PYZhy{}}\PY{n}{N}\PY{o}{*}\PY{n}{mp}\PY{o}{.}\PY{n}{log}\PY{p}{(}\PY{l+m+mi}{2}\PY{o}{*}\PY{n}{mp}\PY{o}{.}\PY{n}{cosh}\PY{p}{(}\PY{n}{beta}\PY{o}{*}\PY{n}{h}\PY{o}{+}\PY{n}{phi}\PY{p}{)}\PY{p}{)}
\end{Verbatim}
\end{tcolorbox}

    input p,phi: ; output: H(p,phi)

    \begin{tcolorbox}[breakable, size=fbox, boxrule=1pt, pad at break*=1mm,colback=cellbackground, colframe=cellborder]
\prompt{In}{incolor}{44}{\boxspacing}
\begin{Verbatim}[commandchars=\\\{\}]
\PY{k}{def} \PY{n+nf}{HMC}\PY{p}{(}\PY{p}{)}\PY{p}{:} \PY{c+c1}{\PYZsh{}Does one iteration of the Markov\PYZhy{}Chain and return phi}
    \PY{k}{global} \PY{n}{N\PYZus{}md}
    \PY{k}{global} \PY{n}{p\PYZus{}0}
    \PY{k}{global} \PY{n}{phi\PYZus{}0}
    \PY{k}{global} \PY{n}{p}
    \PY{k}{global} \PY{n}{ar}
    
    \PY{n}{p} \PY{o}{=} \PY{n}{np}\PY{o}{.}\PY{n}{random}\PY{o}{.}\PY{n}{normal}\PY{p}{(}\PY{n}{loc}\PY{o}{=}\PY{l+m+mf}{0.0}\PY{p}{,} \PY{n}{scale}\PY{o}{=}\PY{l+m+mf}{1.0}\PY{p}{)} 
    
    \PY{n}{p\PYZus{}l}\PY{p}{,}\PY{n}{phi\PYZus{}l} \PY{o}{=} \PY{n}{leapfrog}\PY{p}{(}\PY{n}{p}\PY{p}{,}\PY{n}{phi\PYZus{}0}\PY{p}{)}    
    
    
    \PY{n}{P\PYZus{}acc} \PY{o}{=} \PY{n}{np}\PY{o}{.}\PY{n}{exp}\PY{p}{(}\PY{n+nb}{float}\PY{p}{(}\PY{n}{H}\PY{p}{(}\PY{n}{p}\PY{p}{,}\PY{n}{phi\PYZus{}0}\PY{p}{)}\PY{o}{\PYZhy{}}\PY{n}{H}\PY{p}{(}\PY{n}{p\PYZus{}l}\PY{p}{,}\PY{n}{phi\PYZus{}l}\PY{p}{)}\PY{p}{)}\PY{p}{)}
    
        
    \PY{k}{if} \PY{n}{P\PYZus{}acc} \PY{o}{\PYZgt{}} \PY{n}{np}\PY{o}{.}\PY{n}{random}\PY{o}{.}\PY{n}{rand}\PY{p}{(}\PY{p}{)}\PY{p}{:} 
        
        \PY{n}{phi\PYZus{}0} \PY{o}{=} \PY{n}{phi\PYZus{}l}
        \PY{n}{ar}\PY{o}{=}\PY{n}{ar}\PY{o}{+}\PY{l+m+mi}{1}

    \PY{k}{return} \PY{n}{phi\PYZus{}0}
\end{Verbatim}
\end{tcolorbox}

    Classical HMC-Algo, which returns the next element of the markov chain.
Candidates are created with the leapfrog-algo. In our case it also keeps
track of the acceptance probability with ar.

    \begin{tcolorbox}[breakable, size=fbox, boxrule=1pt, pad at break*=1mm,colback=cellbackground, colframe=cellborder]
\prompt{In}{incolor}{45}{\boxspacing}
\begin{Verbatim}[commandchars=\\\{\}]
\PY{k}{def} \PY{n+nf}{ensemble}\PY{p}{(}\PY{p}{)}\PY{p}{:} \PY{c+c1}{\PYZsh{}Generates an ensemble of phi and counts the acceptance rate}
    \PY{k}{global} \PY{n}{N\PYZus{}en}
    \PY{k}{global} \PY{n}{N\PYZus{}md}
    \PY{k}{global} \PY{n}{p\PYZus{}0}
    \PY{k}{global} \PY{n}{phi\PYZus{}0}
    
    \PY{n}{phi\PYZus{}en\PYZus{}calc} \PY{o}{=}\PY{n}{np}\PY{o}{.}\PY{n}{zeros}\PY{p}{(}\PY{n}{N\PYZus{}cfg}\PY{p}{)}
   
    
    \PY{k}{for} \PY{n}{i} \PY{o+ow}{in} \PY{n+nb}{range}\PY{p}{(}\PY{n}{N\PYZus{}cfg}\PY{p}{)}\PY{p}{:}
        \PY{n}{phi\PYZus{}en\PYZus{}calc}\PY{p}{[}\PY{n}{i}\PY{p}{]} \PY{o}{=} \PY{n}{HMC}\PY{p}{(}\PY{p}{)}

    
    \PY{k}{return} \PY{n}{phi\PYZus{}en\PYZus{}calc}
\end{Verbatim}
\end{tcolorbox}

    This generates the ensemble which is represented by the \(\phi\) which
is stored i an array. The array if filled by looping over the HMC-algo.

    \begin{tcolorbox}[breakable, size=fbox, boxrule=1pt, pad at break*=1mm,colback=cellbackground, colframe=cellborder]
\prompt{In}{incolor}{46}{\boxspacing}
\begin{Verbatim}[commandchars=\\\{\}]
\PY{k}{def} \PY{n+nf}{m\PYZus{}J\PYZus{}5}\PY{p}{(}\PY{p}{)}\PY{p}{:}  
    \PY{k}{global} \PY{n}{h}
    \PY{k}{global} \PY{n}{N\PYZus{}cfg}
    \PY{k}{global} \PY{n}{beta\PYZus{}J}
    \PY{k}{global} \PY{n}{I}
    \PY{k}{global} \PY{n}{A}
    \PY{k}{global} \PY{n}{ar}
    \PY{k}{global} \PY{n}{N} 
    \PY{k}{global} \PY{n}{phi\PYZus{}en}
    \PY{k}{global} \PY{n}{J} 
    \PY{k}{global} \PY{n}{ar}
    \PY{k}{global} \PY{n}{phi\PYZus{}en\PYZus{}temp}
    \PY{k}{global} \PY{n}{N\PYZus{}md}
    
    \PY{n}{E}\PY{o}{=}\PY{l+m+mi}{18} \PY{c+c1}{\PYZsh{}amount of points for the graph}
    \PY{n}{D}\PY{o}{=}\PY{l+m+mi}{4}
    \PY{n}{phi\PYZus{}en}\PY{o}{=}\PY{n}{np}\PY{o}{.}\PY{n}{zeros}\PY{p}{(}\PY{n}{N\PYZus{}cfg}\PY{p}{)}
    \PY{n}{phi\PYZus{}en\PYZus{}temp}\PY{o}{=}\PY{n}{np}\PY{o}{.}\PY{n}{zeros}\PY{p}{(}\PY{p}{(}\PY{n}{E}\PY{o}{*}\PY{n}{D}\PY{p}{,}\PY{n}{N\PYZus{}cfg}\PY{p}{)}\PY{p}{)}
    \PY{n}{colours}\PY{o}{=}\PY{p}{[}\PY{l+s+s1}{\PYZsq{}}\PY{l+s+s1}{b}\PY{l+s+s1}{\PYZsq{}}\PY{p}{,}\PY{l+s+s1}{\PYZsq{}}\PY{l+s+s1}{g}\PY{l+s+s1}{\PYZsq{}}\PY{p}{,}\PY{l+s+s1}{\PYZsq{}}\PY{l+s+s1}{r}\PY{l+s+s1}{\PYZsq{}}\PY{p}{,}\PY{l+s+s1}{\PYZsq{}}\PY{l+s+s1}{c}\PY{l+s+s1}{\PYZsq{}}\PY{p}{,}\PY{l+s+s1}{\PYZsq{}}\PY{l+s+s1}{m}\PY{l+s+s1}{\PYZsq{}}\PY{p}{,}\PY{l+s+s1}{\PYZsq{}}\PY{l+s+s1}{y}\PY{l+s+s1}{\PYZsq{}}\PY{p}{,}\PY{l+s+s1}{\PYZsq{}}\PY{l+s+s1}{k}\PY{l+s+s1}{\PYZsq{}}\PY{p}{]}    
   
    \PY{k}{for} \PY{n}{m} \PY{o+ow}{in} \PY{n+nb}{range}\PY{p}{(}\PY{n}{D}\PY{p}{)}\PY{p}{:} \PY{c+c1}{\PYZsh{}different lattice sizes      }
        \PY{n}{N}\PY{o}{=}\PY{p}{(}\PY{n}{m}\PY{o}{+}\PY{l+m+mi}{1}\PY{p}{)}\PY{o}{*}\PY{l+m+mi}{5}
        \PY{n}{final\PYZus{}M}\PY{o}{=}\PY{n}{np}\PY{o}{.}\PY{n}{zeros}\PY{p}{(}\PY{n}{E}\PY{p}{)}
        \PY{n}{ar}\PY{o}{=}\PY{l+m+mi}{0}
        \PY{n}{theo\PYZus{}values} \PY{o}{=} \PY{p}{[}\PY{p}{]}
        \PY{n}{N\PYZus{}md}\PY{o}{=}\PY{p}{(}\PY{n}{m}\PY{o}{+}\PY{l+m+mi}{1}\PY{p}{)}\PY{o}{*}\PY{l+m+mi}{7}
        \PY{k}{for} \PY{n}{j} \PY{o+ow}{in} \PY{n+nb}{range}\PY{p}{(}\PY{n}{E}\PY{p}{)}\PY{p}{:} \PY{c+c1}{\PYZsh{}Calculation for different J}
            \PY{n}{J}\PY{o}{=}\PY{p}{(}\PY{l+m+mf}{0.2}\PY{o}{+}\PY{p}{(}\PY{n}{j}\PY{o}{/}\PY{l+m+mi}{10}\PY{p}{)}\PY{p}{)}\PY{o}{/}\PY{n}{N}     \PY{c+c1}{\PYZsh{} J=beta*J \PYZbs{}in [0.2, 2.2) \PYZsh{}J sollte in einheiten 1/N sein}
            \PY{n}{phi\PYZus{}en} \PY{o}{=} \PY{n}{ensemble}\PY{p}{(}\PY{p}{)}
            \PY{n}{phi\PYZus{}en\PYZus{}temp}\PY{p}{[}\PY{n}{j}\PY{o}{+}\PY{p}{(}\PY{n}{m}\PY{o}{*}\PY{n}{E}\PY{p}{)}\PY{p}{]}\PY{o}{=}\PY{n}{phi\PYZus{}en}
            \PY{n}{final\PYZus{}M}\PY{p}{[}\PY{n}{j}\PY{p}{]} \PY{o}{=} \PY{n}{calc\PYZus{}magn}\PY{p}{(}\PY{n}{phi\PYZus{}en}\PY{p}{)}
        \PY{n}{J\PYZus{}x}\PY{o}{=}\PY{n}{np}\PY{o}{.}\PY{n}{arange}\PY{p}{(}\PY{l+m+mf}{0.2}\PY{p}{,}\PY{l+m+mf}{2.0}\PY{p}{,}\PY{l+m+mf}{0.1}\PY{p}{)}\PY{o}{/}\PY{n}{N}
        \PY{k}{for} \PY{n}{J} \PY{o+ow}{in} \PY{n}{J\PYZus{}x}\PY{p}{:}
            \PY{n}{theo\PYZus{}values}\PY{o}{.}\PY{n}{append}\PY{p}{(}\PY{n}{m\PYZus{}theo}\PY{p}{(}\PY{p}{)}\PY{p}{)}
          
          
        \PY{n+nb}{print}\PY{p}{(}\PY{l+s+s2}{\PYZdq{}}\PY{l+s+s2}{Acceptance rate for lattice size }\PY{l+s+s2}{\PYZdq{}}\PY{p}{,} \PY{n}{N}\PY{p}{,}\PY{l+s+s2}{\PYZdq{}}\PY{l+s+s2}{ :}\PY{l+s+s2}{\PYZdq{}}\PY{p}{,} \PY{n}{ar}\PY{o}{/}\PY{p}{(}\PY{n}{N\PYZus{}cfg}\PY{o}{*}\PY{n}{E}\PY{p}{)}\PY{o}{*}\PY{l+m+mi}{100}\PY{p}{,}\PY{l+s+s2}{\PYZdq{}}\PY{l+s+s2}{\PYZpc{}}\PY{l+s+s2}{\PYZdq{}}\PY{p}{)}
        \PY{n}{plt}\PY{o}{.}\PY{n}{plot}\PY{p}{(}\PY{n}{J\PYZus{}x}\PY{o}{*}\PY{o}{*}\PY{p}{(}\PY{o}{\PYZhy{}}\PY{l+m+mi}{1}\PY{p}{)}\PY{p}{,}\PY{n}{final\PYZus{}M}\PY{p}{,}\PY{n}{color}\PY{o}{=}\PY{n}{colours}\PY{p}{[}\PY{n}{m}\PY{o}{+}\PY{l+m+mi}{1}\PY{p}{]}\PY{p}{,}\PYZbs{}
                 \PY{n}{label}\PY{o}{=}\PY{l+s+s1}{\PYZsq{}}\PY{l+s+s1}{\PYZdl{}N=\PYZdl{}}\PY{l+s+si}{\PYZpc{}s}\PY{l+s+s1}{\PYZsq{}}\PY{o}{\PYZpc{}}\PY{k}{str}(N))
        \PY{n}{plt}\PY{o}{.}\PY{n}{plot}\PY{p}{(}\PY{n}{J\PYZus{}x}\PY{o}{*}\PY{o}{*}\PY{p}{(}\PY{o}{\PYZhy{}}\PY{l+m+mi}{1}\PY{p}{)}\PY{p}{,}\PY{n}{theo\PYZus{}values}\PY{p}{,} \PY{n}{color}\PY{o}{=}\PY{n}{colours}\PY{p}{[}\PY{n}{m}\PY{o}{+}\PY{l+m+mi}{1}\PY{p}{]}\PY{p}{,} \PY{n}{linestyle}\PY{o}{=}\PY{l+s+s1}{\PYZsq{}}\PY{l+s+s1}{dashed}\PY{l+s+s1}{\PYZsq{}}\PY{p}{,} \PY{n}{label}\PY{o}{=}\PY{l+s+s1}{\PYZsq{}}\PY{l+s+s1}{Theory for \PYZdl{}N=\PYZdl{}}\PY{l+s+si}{\PYZpc{}s}\PY{l+s+s1}{\PYZsq{}}\PY{o}{\PYZpc{}}\PY{k}{str}(N))
    \PY{n}{plt}\PY{o}{.}\PY{n}{legend}\PY{p}{(}\PY{p}{)}
    \PY{n}{plt}\PY{o}{.}\PY{n}{xlabel}\PY{p}{(}\PY{l+s+sa}{r}\PY{l+s+s2}{\PYZdq{}}\PY{l+s+s2}{\PYZdl{}(J/N)\PYZca{}}\PY{l+s+s2}{\PYZob{}}\PY{l+s+s2}{\PYZhy{}1\PYZcb{}\PYZdl{}}\PY{l+s+s2}{\PYZdq{}}\PY{p}{)}
    \PY{n}{plt}\PY{o}{.}\PY{n}{ylabel}\PY{p}{(}\PY{l+s+sa}{r}\PY{l+s+s2}{\PYZdq{}}\PY{l+s+s2}{\PYZdl{}m((J/N)\PYZca{}}\PY{l+s+s2}{\PYZob{}}\PY{l+s+s2}{\PYZhy{}1\PYZcb{})\PYZdl{}}\PY{l+s+s2}{\PYZdq{}}\PY{p}{)}
    \PY{n}{plt}\PY{o}{.}\PY{n}{show}\PY{p}{(}\PY{p}{)}
\end{Verbatim}
\end{tcolorbox}

    We run through different N and J with for loops. We save all our created
ensembles globally in phi\_en\_temp, so that we dont have to create them
again. We vary our \(N_{md}\) for the different lattice sizes. We plot
the magnetization as well as the theoretically expected magnetization
against \((J/N)^{-1}\).

    \begin{tcolorbox}[breakable, size=fbox, boxrule=1pt, pad at break*=1mm,colback=cellbackground, colframe=cellborder]
\prompt{In}{incolor}{47}{\boxspacing}
\begin{Verbatim}[commandchars=\\\{\}]
\PY{n}{m\PYZus{}J\PYZus{}5}\PY{p}{(}\PY{p}{)}
\end{Verbatim}
\end{tcolorbox}

    \begin{Verbatim}[commandchars=\\\{\}]
Acceptance rate for lattice size  5  : 99.25555555555555 \%
Acceptance rate for lattice size  10  : 99.76666666666667 \%
Acceptance rate for lattice size  15  : 98.9074074074074 \%
Acceptance rate for lattice size  20  : 99.09259259259258 \%
    \end{Verbatim}

    \begin{center}
    \adjustimage{max size={0.9\linewidth}{0.9\paperheight}}{output_28_1.png}
    \end{center}
    { \hspace*{\fill} \\}
    
    One can see that the acceptance probability is very high, which is good
but also costly.\\
The graphs fit the theoretical expectation pretty well. One could only
argue that for large lattice sizes and large \((J/N)^{-1}\) the
calculated value differs a bit from theoretical expectation. This is due
to the fact that our size of the ensemble is fixed and therefor its very
costly to higher the value which would increase the ensemble size for
all configurations.

    \begin{tcolorbox}[breakable, size=fbox, boxrule=1pt, pad at break*=1mm,colback=cellbackground, colframe=cellborder]
\prompt{In}{incolor}{48}{\boxspacing}
\begin{Verbatim}[commandchars=\\\{\}]
\PY{k}{def} \PY{n+nf}{e\PYZus{}J\PYZus{}5}\PY{p}{(}\PY{p}{)}\PY{p}{:}  
    \PY{k}{global} \PY{n}{h}
    \PY{k}{global} \PY{n}{N\PYZus{}cfg}
    \PY{k}{global} \PY{n}{beta\PYZus{}J}
    \PY{k}{global} \PY{n}{I}
    \PY{k}{global} \PY{n}{A}
    \PY{k}{global} \PY{n}{ar}
    \PY{k}{global} \PY{n}{N} 
    \PY{k}{global} \PY{n}{phi\PYZus{}en}
    \PY{k}{global} \PY{n}{J} 
    \PY{k}{global} \PY{n}{ar}
    \PY{k}{global} \PY{n}{phi\PYZus{}en\PYZus{}temp}
 
    
    \PY{n}{E}\PY{o}{=}\PY{l+m+mi}{18} \PY{c+c1}{\PYZsh{}amount of points for the graph}
    \PY{n}{D}\PY{o}{=}\PY{l+m+mi}{4}
    
    
    \PY{n}{colours}\PY{o}{=}\PY{p}{[}\PY{l+s+s1}{\PYZsq{}}\PY{l+s+s1}{b}\PY{l+s+s1}{\PYZsq{}}\PY{p}{,}\PY{l+s+s1}{\PYZsq{}}\PY{l+s+s1}{g}\PY{l+s+s1}{\PYZsq{}}\PY{p}{,}\PY{l+s+s1}{\PYZsq{}}\PY{l+s+s1}{r}\PY{l+s+s1}{\PYZsq{}}\PY{p}{,}\PY{l+s+s1}{\PYZsq{}}\PY{l+s+s1}{c}\PY{l+s+s1}{\PYZsq{}}\PY{p}{,}\PY{l+s+s1}{\PYZsq{}}\PY{l+s+s1}{m}\PY{l+s+s1}{\PYZsq{}}\PY{p}{,}\PY{l+s+s1}{\PYZsq{}}\PY{l+s+s1}{y}\PY{l+s+s1}{\PYZsq{}}\PY{p}{,}\PY{l+s+s1}{\PYZsq{}}\PY{l+s+s1}{k}\PY{l+s+s1}{\PYZsq{}}\PY{p}{]}    
   
    \PY{k}{for} \PY{n}{m} \PY{o+ow}{in} \PY{n+nb}{range}\PY{p}{(}\PY{n}{D}\PY{p}{)}\PY{p}{:} \PY{c+c1}{\PYZsh{}different lattice sizes      }
        \PY{n}{N}\PY{o}{=}\PY{p}{(}\PY{n}{m}\PY{o}{+}\PY{l+m+mi}{1}\PY{p}{)}\PY{o}{*}\PY{l+m+mi}{5}
        \PY{n}{final\PYZus{}e}\PY{o}{=}\PY{n}{np}\PY{o}{.}\PY{n}{zeros}\PY{p}{(}\PY{n}{E}\PY{p}{)}
        \PY{n}{ar}\PY{o}{=}\PY{l+m+mi}{0}
        \PY{n}{theo\PYZus{}values} \PY{o}{=} \PY{p}{[}\PY{p}{]}
        \PY{k}{for} \PY{n}{j} \PY{o+ow}{in} \PY{n+nb}{range}\PY{p}{(}\PY{n}{E}\PY{p}{)}\PY{p}{:} \PY{c+c1}{\PYZsh{}Calculation for different J}
            \PY{n}{J}\PY{o}{=}\PY{p}{(}\PY{l+m+mf}{0.2}\PY{o}{+}\PY{p}{(}\PY{n}{j}\PY{o}{/}\PY{l+m+mi}{10}\PY{p}{)}\PY{p}{)}\PY{o}{/}\PY{n}{N}    
            \PY{n}{phi\PYZus{}en}\PY{o}{=}\PY{n}{phi\PYZus{}en\PYZus{}temp}\PY{p}{[}\PY{n}{j}\PY{o}{+}\PY{p}{(}\PY{n}{m}\PY{o}{*}\PY{n}{E}\PY{p}{)}\PY{p}{]}
            \PY{n}{final\PYZus{}e}\PY{p}{[}\PY{n}{j}\PY{p}{]} \PY{o}{=} \PY{n}{calc\PYZus{}eps}\PY{p}{(}\PY{n}{phi\PYZus{}en}\PY{p}{)}
        \PY{n}{J\PYZus{}x}\PY{o}{=}\PY{n}{np}\PY{o}{.}\PY{n}{arange}\PY{p}{(}\PY{l+m+mf}{0.2}\PY{p}{,}\PY{l+m+mf}{2.0}\PY{p}{,}\PY{l+m+mf}{0.1}\PY{p}{)}\PY{o}{/}\PY{n}{N}
        \PY{k}{for} \PY{n}{J} \PY{o+ow}{in} \PY{n}{J\PYZus{}x}\PY{p}{:}
            \PY{n}{theo\PYZus{}values}\PY{o}{.}\PY{n}{append}\PY{p}{(}\PY{n}{beta\PYZus{}eps\PYZus{}theo}\PY{p}{(}\PY{p}{)}\PY{p}{)}
        \PY{n}{plt}\PY{o}{.}\PY{n}{plot}\PY{p}{(}\PY{n}{J\PYZus{}x}\PY{o}{*}\PY{o}{*}\PY{p}{(}\PY{o}{\PYZhy{}}\PY{l+m+mi}{1}\PY{p}{)}\PY{p}{,}\PY{n}{final\PYZus{}e}\PY{p}{,}\PY{n}{color}\PY{o}{=}\PY{n}{colours}\PY{p}{[}\PY{n}{m}\PY{o}{+}\PY{l+m+mi}{1}\PY{p}{]}\PY{p}{,}\PYZbs{}
                 \PY{n}{label}\PY{o}{=}\PY{l+s+s1}{\PYZsq{}}\PY{l+s+s1}{\PYZdl{}N=\PYZdl{}}\PY{l+s+si}{\PYZpc{}s}\PY{l+s+s1}{\PYZsq{}}\PY{o}{\PYZpc{}}\PY{k}{str}(N))
        \PY{n}{plt}\PY{o}{.}\PY{n}{plot}\PY{p}{(}\PY{n}{J\PYZus{}x}\PY{o}{*}\PY{o}{*}\PY{p}{(}\PY{o}{\PYZhy{}}\PY{l+m+mi}{1}\PY{p}{)}\PY{p}{,}\PY{n}{theo\PYZus{}values}\PY{p}{,} \PY{n}{color}\PY{o}{=}\PY{n}{colours}\PY{p}{[}\PY{n}{m}\PY{o}{+}\PY{l+m+mi}{1}\PY{p}{]}\PY{p}{,} \PY{n}{linestyle}\PY{o}{=}\PY{l+s+s1}{\PYZsq{}}\PY{l+s+s1}{dashed}\PY{l+s+s1}{\PYZsq{}}\PY{p}{,} \PY{n}{label}\PY{o}{=}\PY{l+s+s1}{\PYZsq{}}\PY{l+s+s1}{Theory for \PYZdl{}N=\PYZdl{}}\PY{l+s+si}{\PYZpc{}s}\PY{l+s+s1}{\PYZsq{}}\PY{o}{\PYZpc{}}\PY{k}{str}(N))
    \PY{n}{plt}\PY{o}{.}\PY{n}{legend}\PY{p}{(}\PY{p}{)}
    \PY{n}{plt}\PY{o}{.}\PY{n}{xlabel}\PY{p}{(}\PY{l+s+sa}{r}\PY{l+s+s2}{\PYZdq{}}\PY{l+s+s2}{\PYZdl{}(J/N)\PYZca{}}\PY{l+s+s2}{\PYZob{}}\PY{l+s+s2}{\PYZhy{}1\PYZcb{}\PYZdl{}}\PY{l+s+s2}{\PYZdq{}}\PY{p}{)}
    \PY{n}{plt}\PY{o}{.}\PY{n}{ylabel}\PY{p}{(}\PY{l+s+sa}{r}\PY{l+s+s2}{\PYZdq{}}\PY{l+s+s2}{\PYZdl{}}\PY{l+s+s2}{\PYZbs{}}\PY{l+s+s2}{beta}\PY{l+s+s2}{\PYZbs{}}\PY{l+s+s2}{cdot }\PY{l+s+s2}{\PYZbs{}}\PY{l+s+s2}{epsilon(J/N)\PYZca{}}\PY{l+s+s2}{\PYZob{}}\PY{l+s+s2}{\PYZhy{}1\PYZcb{}\PYZdl{}}\PY{l+s+s2}{\PYZdq{}}\PY{p}{)}
    \PY{n}{plt}\PY{o}{.}\PY{n}{show}\PY{p}{(}\PY{p}{)}
\end{Verbatim}
\end{tcolorbox}

    We run through different N and J with for loops. We use the saved
ensembles in phi\_en\_temp. We plot the energy per site as well as the
theoretically expected energy per site against \((J/N)^{-1}\).

    \begin{tcolorbox}[breakable, size=fbox, boxrule=1pt, pad at break*=1mm,colback=cellbackground, colframe=cellborder]
\prompt{In}{incolor}{49}{\boxspacing}
\begin{Verbatim}[commandchars=\\\{\}]
\PY{n}{e\PYZus{}J\PYZus{}5}\PY{p}{(}\PY{p}{)}
\end{Verbatim}
\end{tcolorbox}

    \begin{center}
    \adjustimage{max size={0.9\linewidth}{0.9\paperheight}}{output_32_0.png}
    \end{center}
    { \hspace*{\fill} \\}
    
    The energy per site also matches the expected values very well. As well
as above, there are small deviations for large lattice sizes and large
\((J/N)^{-1}\). This is to be expected, because our calculations are
based on the same ensembles.


    % Add a bibliography block to the postdoc
    
    
    
\end{document}
